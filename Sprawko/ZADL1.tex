\chapter{Pomiar w punkcie pracy}
\section{Komunikacja z obiektem}
Komunikacja z obiektem odbywa się za pomocą funkcji napisanych w środowisku MatLab. Najprostszy program prezentujący sposób ich użycia znajduje się poniżej.
%\begin{lstlisting}[style=custommatlab,frame=single]
\begin{lstlisting}[style=custommatlab]

addpath('F:\SerialCommunication'); % add a path to the functions
initSerialControl COM3 % initialise com port
N = 360;
i = 0;
while(i<=N)
	%% obtaining measurements
	measurements = readMeasurements(1:7); 
	
	%% processing of the measurements
	disp(measurements(1));
	
	%% sending new values of control signals
	sendControlsToG1AndDisturbance(29,10);
	
	i=i+1;
	% wait for new batch of measurements to be ready
	waitForNewIteration();
end
\end{lstlisting} 
W tym zadaniu uzywamy funkcji sendControlsToG1AndDisturbance, która pozwala wysyłać sterowanie na grzałkę pierwszą oraz wartość zakłóczenia. Funkcja oraz cały skrypt działą poprawnie.
\section{Punkt pracy}
Sygnałem sterującym jest moc grzania pierwszą grzałką, natomiast zakłóceniem jest nieznanym wzmocnieniem tej grzałki. Pierwszy wentylator (1 element sterowalny) ma być na stałe ustawiony na 50\% (cecha otoczenia), a pozostałe elementy mają pozostać wyłączone. Wyjsciem ma być temperatura zmierzona przez czujnik temperatury T1.
Wartoscią sygnału sterującego w naszym punkcie pracy ma być 29\%, natomiast wartość zakłóceń ma być równa 0.
Ustabilizowana wartość wyjścia obiektu  odpowiada temperaturze na T1 równej \num{32.75}. Warto zwrócić uwagę na to, iż znacząco się ona różni (o około dwa stopnie) w stosunku do analogicznego punktu pracy w pierwszym zadaniu. Wpływ na to mają nieuwzględniane zakłócenia, jak np. temperatura powietrza w sali.